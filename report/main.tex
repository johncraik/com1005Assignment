% ---------------------------- %
% Assignment Report Template
% 
% Heidi Christensen, 2020
% --------------------------   %


\documentclass[11pt,oneside]{article}
\usepackage[utf8]{inputenc}
\usepackage{graphicx}


\title{Experimental report for the 2021 COM1005 Assignment: The Rambler's Problem\footnote{https://github.com/johncraik/com1005Assignment.git}}
\author{John Craik}
\date{\today}


\begin{document}

\maketitle

\section{Description of my branch-and-bound implementation}
To implement branch-and-bound, I created two classes that would extend existing classes that had been provided. These where as follows;
\begin{itemize}
    \item The class 'RamblersSearch', which extended the provided class 'Search', was used to define the map and goal co-ordinate.
    \item The class 'RamblersState', which extended the provided class 'SearchState', was used to calculate the successor states and their corresponding costs.
\end{itemize}
The 'getSuccessors' function in the class 'RamblersState' was used to get the possible successor states from the current state. The maps used in the search are pgm maps, so each value in the pgm is connected to the values left, right, top and bottom to the value (if such values exist to the left, right, top and bottom). This gave up to 4 possible successors to each state.

The cost of moving from one state to another was dependent on the height of the current state and the height of the successor state. If the successor state had a height that was lower or equal to the current states height, the cost would simply be 1. However, if the successor state has a height that was greater than the height of the  current states height, the cost was defined by this equation:
\begin{itemize}
    \item \textbf{Cost = 1 + (successorHeight(y,x) - currentHeight(y,x)).}
\end{itemize}
The heights were obtained from a 2D array representing the pgm map, and then compared with eachother to calculate the cost, adding each possible successor state into a list that would be returned by this function.


\section{Description of my A* implementation}
Implementing the A* algorithm was fairly simple as most of the work had already been done when implementing branch and bound (however, the code provided in 'search4' was used, not 'search3'). The only difference is the calculation of the cost, as A* uses both ramblers cost and an estimated remaining cost, which is based on the distance the current state is from the goal state.


\section{Assessing efficiency}
 

\subsection{Assessing the efficiency of my branch-and-bound search algorithm}
Table 1 shows 30 randomly chosen start and goal states and their corresponding efficiency, from a 100x100 pgm map. The data below shows the mean average, highest and lowest:
\begin{itemize}
	\item Average Efficiency: 0.03197
	\item Hieghst Efficiency: 0.1818
	\item Lowest Efficiency: 0.00843
	\item Range: 0.17337
\end{itemize}

\begin{table}[ht]
    \centering
    \begin{tabular}{|c|c|c|}
	Start State (y,x)&Goal State (y,x)&Efficiency (4sf) \\ \hline
	(90,70)&(75,38)&0.02169 \\
	(8,32)	&(93,7)&0.02222 \\
	(21,68)&(29,80)&0.01599 \\
	(17,88)&(61,2)&0.02311 \\
	(86,49)&(65,41)&0.01444 \\
	(3,87)&(92,98)&0.01804 \\
	(79,80)&(24,79)&0.01455 \\
	(25,20)&(64,77)&0.01479 \\
	(69,70)&(48,46)&0.01545 \\
	(36,65)&(42,91)&0.02798 \\
	(52,39)&(4,27)&0.01227 \\
	(2,47)&(18,22)&0.02340 \\
	(14,80)&(87,76)&0.01926 \\
	(9,52)&(19,39)&0.01199 \\
	(52,1)&(20,34)&0.01360 \\
	(60,0)&(97,84)&0.01492 \\
	(56,60)&(1,50)&0.00843 \\
	(54,88)&(87,83)&0.01060 \\
	(56,4)&(61,33)&0.07559 \\
	(82,55)&(86,57)&0.09465 \\
	(91,96)&(50,5)&0.02513 \\
	(83,34)&(94,88)&0.02283 \\
	(41,92)&(36,40)&0.05949 \\
	(62,76)&(24,59)&0.00923 \\
	(54,75)&(19,0)&0.01596 \\
	(34,88)&(70,75)&0.01059 \\
	(27,98)&(68,15)&0.01801 \\
	(76,7)&(79,35)&0.1096 \\
	(44,60)&(43,54)&0.1818 \\
	(76,87)&(87,5)&0.03351 \\

    \end{tabular}
    \caption{The effieciencies of 30 random start and goal states on a 100x100 pgm map, using branch-and-bound.}
    \label{tab:bandbeff}
\end{table}


\subsection{Assessing the efficiency of my A* search algorithm}
Table 2 shows the efficencies of the same 30 states when using the A* algortithm The data below shows the mean average, highest and lowest:
\begin{itemize}
	\item Average Efficiency: 0.07348
	\item Hieghst Efficiency: 0.4375
	\item Lowest Efficiency: 0.01848
	\item Range: 0.41902
\end{itemize}

\begin{table}[ht]
    \centering
    \begin{tabular}{|c|c|c|}
	Start State (y,x)&Goal State (y,x)&Efficiency (4sf) \\ \hline
	(90,70)&(75,38)&0.06409 \\
	(8,32)&(93,7)&0.02834 \\
	(21,68)&(29,80)&0.168 \\
	(17,88)&(61,2)&0.02039 \\
	(86,49)&(65,41)&0.07633 \\
	(3,87)&(92,98)&0.02260 \\
	(79,80)&(24,79)&0.1691 \\
	(25,20)&(64,77)&0.02066 \\
	(69,70)&(48,46)&0.1055 \\
	(36,65)&(42,91)&0.09621 \\
	(52,39)&(4,27)&0.04111 \\
	(2,47)&(18,22)&0.03097 \\
	(14,80)&(87,76)&0.01848 \\
	(9,52)&(19,39)&0.01848 \\
	(52,1)&(20,34)&0.04885 \\
	(60,0)&(97,84)&0.02029 \\
	(56,60)&(1,50)&0.1514 \\
	(54,88)&(87,83)&0.03380 \\
	(56,4)	&(61,33)&0.03380 \\
	(82,55)&(86,57)&0.4375 \\
	(91,96)&(50,5)&0.01910 \\
	(83,34)&(94,88)&0.03411 \\
	(41,92)&(36,40)&0.03411 \\
	(62,76)&(24,59)&0.1290 \\
	(54,75)&(19,0)&0.02027 \\
	(34,88)&(70,75)&0.02027\\
	(27,98)&(68,15)&0.01918 \\
	(76,7)	&(79,35)&0.04604 \\
	(44,60)&(43,54)&0.2051 \\
	(76,87)&(87,5)&0.07127 \\

    \end{tabular}
    \caption{The effieciencies of the same 30 start and goal states used in branch-and-bound, however searching using the A* algorithm.}
    \label{tab:astareff}
\end{table}

\subsection{Comparing the two search strategies}
\textit{Still to implement.}

\section{Conclusions}
\textit{Still to implement.}

\end{document}
